%!TEX root=main.tex

\subsection{Testes ???}

As estatísticas apresentadas são em relação ao jogador destacado a negrito. As vitórias, derrotas, empates e número médio de turnos são representados respetivamente pelas cores azul, vermelha, verde e preta. As três primeiras são apresentadas em permilagem. Todos valores originais foram arredondados para baixo pelo que o somatório das componentes pode n\~ao igualar 1000\perthousand. 

O jogador artificial que jogar com as brancas inicia a partida (ao contrário do Pentago original). 

As informações apresentadas abaixo do nome das heurísticas referem-se aos pesos/opç\~oes usadas. Nas heurísticas '1' e '1.2' são sempre usados (e ocultados nas tabelas) os pesos \verb|default| usados para os diferentes tipos de linhas, pois a sua inclusão nos testes não se justifica, dada a reduzida relevância para projeto tendo em conta os dados já apresentados e o aumento considerável de casos a testar, Muitas das combinações usadas foram descobertas ao testar, de forma automatizada, as heurísticas com valores pseudo-aleatórios (dentro de determinados intervalos).

Para cada cada heurística os dados apresentados referem-se a:
\begin{itemize}  

\item '1' - \verb|bias|
\item '1.2' Relaxada? (Y=sim,N=não); Usa Hack Diagonal? (Y=sim,N=não); possibilidades ; possibilidades para oponente; possibilidades fortes; possibilidades fortes para oponente
\item 'A' - monica ; meio ; reta ; tripla
\item 'A hacked' - Pesos da '1.2' ; Pesos da 'A'
\item 'A start' - Pesos da 'A' ;  Analisa Rotaç\~oes? (Y=sim,N=não)
\end{itemize} 

A primeira ocorrência de uma heurística numa tabela apresenta sempre os pesos \verb|default| da mesma.  

\subsubsection{Testes de Heurísticas versus Controlo}

Cada teste consiste em 400 partidas com tabuleiro inicial vazio e 400 com tabuleiros aleatórios. Dos 400 tabuleiros aleatórios uma metade são com um número de peças ímpares (jogador com pretas a iniciar a primeira jogada) e a outra metade com um número de peças par (jogador com brancas a iniciar a primeira jogada). 

\input{tables/WDL_controlonly.tex}
\input{tables/WDL_VScontrol.tex}

\newpage
\subsubsection{Testes de Heurísticas versus Heurísticas}

Cada teste consiste em 200 partidas com tabuleiro inicial vazio e 200 com tabuleiros aleatórios. Dos 200 tabuleiros aleatórios uma metade são com um número de peças ímpares (jogador com pretas a iniciar a primeira jogada) e a outra metade com um número de peças par (jogador com brancas a iniciar a primeira jogada).

\input{tables/WDL_IAs.tex}

\input{tables/WDL_others.tex}
